%%%%%%%%%%%%%%%%%%%%%%%%%%%%%%%%%%%%%%%%%
% Friggeri Resume/CV
% XeLaTeX Template
% Version 1.2 (3/5/15)
%
% This template has been downloaded from:
% http://www.LaTeXTemplates.com
%
% Original author:
% Adrien Friggeri (adrien@friggeri.net)
% https://github.com/afriggeri/CV
%
% License:
% CC BY-NC-SA 3.0 (http://creativecommons.org/licenses/by-nc-sa/3.0/)
%
% Important notes:
% This template needs to be compiled with XeLaTeX and the bibliography, if used,
% needs to be compiled with biber rather than bibtex.
%
%%%%%%%%%%%%%%%%%%%%%%%%%%%%%%%%%%%%%%%%%

\documentclass[]{friggeri-cv} % Add 'print' as an option into the square bracket to remove colors from this template for printing

\usepackage{polski}
\usepackage[polish,english]{babel}
\usepackage[utf8]{inputenc}
%\usepackage[T1]{fontenc}
\usepackage{aas_macros}
\addbibresource{bibliography.bib} % Specify the bibliography file to include publications


\begin{document}

\header{Michał}{Bejger}{}{} % Your name and current job title/field

%----------------------------------------------------------------------------------------
%	SIDEBAR SECTION
%----------------------------------------------------------------------------------------

\begin{aside} % In the aside, each new line forces a line break
\section{Scientific areas} 
Data analysis and detection of gravitational waves, machine learning, dense matter equation of state, numerical simulations of relativistic compact objects, high-performance computing.   
~
\section{Contact}
{\bf Nicolaus Copernicus Astronomical Center} 
%Polish Academy of Sciences
ul. Bartycka 18
00-716 Warsaw 
Poland
~
+42 (22) 32 96 130
~
\href{mailto:bejger@camk.edu.pl}{bejger@camk.edu.pl}
\href{http://users.camk.edu.pl/bejger}{users.camk.edu.pl/bejger}
~
\section{Languages {\small (CERF scale)}}
English (C2), German (B2), French (A2)
%\section{programming}
%{\color{red} $\varheartsuit$} JavaScript
%Python, C++, PHP
%CSS3 \& HTML5
\section{Bibliometry} 
{\small (26 April 2020)}
Citations: 30113
h: 58
\href{https://ui.adsabs.harvard.edu/search/q=author\%3A\%22Bejger\%2C\%20M.\%22&sort=date\%20desc\%2C\%20bibcode\%20desc&p_=0}{SAO/NASA ADS}
%\href{http://adsabs.harvard.edu/cgi-bin/nph-abs_connect?db_key=AST&db_key=PHY&db_key=PRE&qform=AST&arxiv_sel=astro-ph&arxiv_sel=cond-mat&arxiv_sel=cs&arxiv_sel=gr-qc&arxiv_sel=hep-ex&arxiv_sel=hep-lat&arxiv_sel=hep-ph&arxiv_sel=hep-th&arxiv_sel=math&arxiv_sel=math-ph&arxiv_sel=nlin&arxiv_sel=nucl-ex&arxiv_sel=nucl-th&arxiv_sel=physics&arxiv_sel=quant-ph&arxiv_sel=q-bio&sim_query=YES&ned_query=YES&adsobj_query=YES&aut_logic=OR&obj_logic=OR&author=bejger\%2C+m.&object=&start_mon=&start_year=&end_mon=&end_year=&ttl_logic=OR&title=&txt_logic=OR&text=&nr_to_return=200&start_nr=1&jou_pick=ALL&ref_stems=&data_and=ALL&group_and=ALL&start_entry_day=&start_entry_mon=&start_entry_year=&end_entry_day=&end_entry_mon=&end_entry_year=&min_score=&sort=SCORE&data_type=SHORT&aut_syn=YES&ttl_syn=YES&txt_syn=YES&aut_wt=1.0&obj_wt=1.0&ttl_wt=0.3&txt_wt=3.0&aut_wgt=YES&obj_wgt=YES&ttl_wgt=YES&txt_wgt=YES&ttl_sco=YES&txt_sco=YES&version=1}{SAO/NASA ADS}
\end{aside}

%----------------------------------------------------------------------------------------
%	EDUCATION SECTION
%----------------------------------------------------------------------------------------

\section{Education}
\begin{entrylist}
%------------------------------------------------
\entry
{2013}
{Habilitation}
{Nicolaus Copernicus Astronomical Center, PAS} 
{{ \it ''Astrophysical parameters of neutron stars as tests of the dense matter properties''} (25.10.2013)} 

\entry
{2001--2005}
{PhD {\normalfont in physics (astrophysics)}}
{Nicolaus Copernicus Astronomical Center, PAS}
{{\it ''Neutron stars dynamics and the equation of state of dense matter''}. Supervisor: Paweł Haensel (16.06.2005; with distinction from the NCAC Scientific Council).} 

\entry
{1996--2001}
{Master {\normalfont of Science}}
{Warsaw University, Faculty of Physics}
%------------------------------------------------

\end{entrylist}

%----------------------------------------------------------------------------------------
%	WORK EXPERIENCE SECTION
%----------------------------------------------------------------------------------------

\section{Positions}
\subsection{Current}
\begin{entrylist}

\entry
{2014--2021}
{Associate professor}
{Nicolaus Copernicus Astronomical Center, PAS, Warsaw, Poland}

\end{entrylist}

\subsection{Previous}

\begin{entrylist}

\entry
{2018--2019}
{Researcher}
{AstroParticule et Cosmologie (APC), CNRS, Paris, France}

\entry
{2008--2014}
{Assistant professor}
{Nicolaus Copernicus Astronomical Center, PAS, Warsaw, Poland}

\entry
{2007--2008}
{Post-doc}
{Nicolaus Copernicus Astronomical Center, PAS, Warsaw, Poland}

\entry
{2006-2007}
{Marie Currie Fellow post-doc}
{Observatoire de Paris, LUTH, Paris-Meudon, France}

\end{entrylist}

%----------------------------------------------------------------------------------------
%	AWARDS SECTION
%----------------------------------------------------------------------------------------

\section{Fellowships and awards}

\begin{entrylistshort}

\entrys 
{10.10.2016}
{W. Rubinowicz Science Prize from Polish Physical Society for the discovery of gravitational waves}  

\entrys 
{04.05.2016}
{Gruber Cosmology Prize, Gruber Prize foundation, for the discovery of gravitational waves}  

\entrys 
{02.05.2016}
{Special Breakthrough Prize in fundamental physics for the authors of the first direct detection of gravitational waves}  

\entrys
{15.03.2016} 
{Nicolaus Copernicus Medal of the Polish Academy of Sciences (for members of the Virgo-POLGRAW team)}   

\entrys
{09--11.2015} 
{DAAD Research Stay for University Academics and Scientists (Steinbuch Centre for Computing, Karlsruhe Institute of Technology, Germany)}  

\entrys 
{04.2008--03.2011} 
{Marie Curie Re-integration Fellowship (NCAC, Warsaw, Poland)}   

\entrys 
{03.2006--08.2007} 
{Marie Curie Intra-European Fellowship (LUTH, Paris, France)} 

\end{entrylistshort}

%\pagebreak 
%\begin{entrylistshort} 

%\entrys 
%{2005} 
%{Honorable mention in prof. G. Białkowski PAS (Foundation for Polish Science) competition for the best PhD thesis in physics/astronomy in years 2004-2006.} 

%\end{entrylistshort}

%\vskip 0.5cm 
% Teaching 

\pagebreak

\setlength{\voffset}{0pt}
\begin{aside} 
\section{Peer review service}
~ 
AAS \& APS journals ({\apj}, {\apjl}, {\prd}, {\prl}), 
{\mnras}, {\aap}, EPJA, MLST, General Relativity and Gravitation, NWA 
~
~ 
% Institutional responsibilities
\section{Institutional responsibilities}
~
2014--2018: Proceedings of the Polish Astronomical Society editor
~
2009--present: Member of the Scientific Council, NCAC 
~
2008--2012: Institute Journal Club host, NCAC
\end{aside} 


%  
\section{Invited talks}
\begin{entrylistoc}

\entrys 
{25.04.2019} 
{{PHAROS 2019}, \href{https://indico.ice.csic.es/event/12/page/12-final-program}{''GW170817: lessons from the observations of a binary neutron star merger''}, {Platja d'Aro, Spain}}  

\entrys 
{25.02.2019} 
{{GWEOS workshop}, \href{https://agenda.infn.it/event/17643/timetable/\#20190225}{''Isolated NS: results and perspectives''}, {Pisa, Italy}} 

\entrys 
{09.10.2018} 
{{Black Hole Initiative seminar}, \href{https://arxiv.org/abs/1704.05931}{''Collisions of neutron stars with primordial black holes as fast radio bursts engines''}, {Harvard Cambridge, USA}} 

\entrys
{12.06.2018}
{{Workshop \href{https://mode2018.sciencesconf.org/program}{''Neutron stars and their environments'' (MODE-SNR-PWN)}}, {''Equation of state and the tidal deformability from gravitational wave measurements of LIGO and Virgo'', Montpelier, France}}

\entrys 
{10.10.2017} 
{{ECT* workshop ''New perspectives on Neutron Star Interiors''},{''Testing relativity with gravitational waves'', Trento, Italy}} 

\entrys
{06.07.2017} 
{\href{https://cosmo.torun.pl/cosmotorun17.html}{Inhomogeneous Cosmologies workshop}, ''Sage Manifolds: differential geometry with SageMath'', Torun, Poland} 

\entrys 
{23.06.2017}
{''Computational challenges of gravitational-wave searches'', \href{http://gpuday.com}{GPU Days 2017, The Future of Many-Core Computing in Science}, Budapest, Hungary} 

\entrys 
{31.03.2017}
{''Review on the continous gravitational wave searches'', \href{http://moriond.in2p3.fr/grav/2017/program.php\#Friday}{Rencontres de Moriond (Gravitation)}, La Thuile, Italy}

\entrys 
{01.12.2016}
{''The first detections of gravitational waves from binary black holes'', \href{http://indico.fuw.edu.pl/sessionDisplay.py?contribId=57\&sessionId=44\&confId=46\#20161201}{DISCRETE 2016} (Special Session of the DISCRETE 2016 Symposium and the Leopold Infeld Colloquium), Warsaw, Poland}

\entrys 
{08.06.2016} 
{''Pierwsza bezposrednia obserwacja fal grawitacyjnych'', General meeting of the Warsaw Scientific Society, Warsaw, Poland} 

\entrys
{26.11.2015}
{''POLGRAW all-sky search for almost monochromatic gravitational waves in the Virgo and LIGO data'', Polish Society on Relativity, Warsaw, Poland}

%\entrys 
%{03.07.2014}
%{''Searches for gravitational waves from known pulsars in the LIGO and Virgo data'', 1st Conference of Polish Society on Relativity, Spała}

%\entrys
%{07.01.2014}
%{''Energetic particle collisions near black-hole horizons'', Israeli-Polish Meeting on Astrophysics, Tel Aviv, Israel}

%\entrys 
%{08.07.2013}
%{''Searching for gravitational wave signals from rotating neutron stars with the LIGO and Virgo detectors'', GR20/Amaldi10, Warsaw, Poland}

\end{entrylistoc}

% 
\vskip 0.5cm 
\section{Leader roles in research grants} 
\begin{entrylistoc}

\entrys 
{2018-2022} 
{Management Committee Member and Work Group Leader in the COST Action ''A network for Gravitational Waves, Geophysics and Machine Learning'', funding: EU Horizon2020 ({\tt COST Action CA17137}) } 
   
\entrys
{2018--2021}
{PI at NCAC in ''Gravitational-wave astronomy: participation of the Polgraw group in Advanced Virgo and Advanced LIGO projects'' HARMONIA project, funding: NCN ({\tt 2017/26/M/ST9/00978})}

\entrys
{2017--2021} 
{PI in ''Transient gravitational waves from neutron stars: models and data analysis'' SONATA BIS project, funding: NCN ({\tt 2016/22/E/ST9/00037})} 

\entrys
{2015--2018} 
{PI at NCAC in ''Participation of Poland in the Advanced Virgo project'' HARMONIA project, funding: NCN ({\tt 2014/14/M/ST9/00707})} 

\entrys
{2013--2017}
{PI at NCAC in ''Networking and R\&D for Einstein Telescope'', funding: NCN/ASPERA Eranet ({\tt 2013/01/ASPERA/ST9/00001})}

\entrys
{2013--2014}
{PI in ''Search for gravitational waves from rotating neutron stars using hardware accelerators'' OPUS project, funding: NCN ({\tt 2012/07/B/ST9/04420})} 

\end{entrylistoc}

\pagebreak 
\printbibsection{article}{10 recent selected publications} % Print all articles from the bibliography

\pagebreak
\section{Teaching}
\begin{entrylistshort}

\entrys
{8--22.07.2017}
{\href{http://cosmoschool2018.oa.uj.edu.pl}{4th Cosmology School: Introduction to cosmology} lecturer, ''Cosmology with Gravitational Waves'', Kraków, Poland}

\entrys
{17.07.2017} 
{Helmholtz International Summer School \href{http://theor.jinr.ru/~ntaa/17}{''Nuclear theory and astrophysical applications''} lecturer, ''Gravitational waves from neutron stars in the era of Advanced LIGO and Advanced Virgo detectors'', Dubna, Russia}  

\entrys
{24--28.10.2016}
{\href{https://events.ego-gw.it/indico/conferenceDisplay.py?ovw=True\&confId=44}{Fifth GraWIToN School} (GW Initial Training Network) lecturer, ''Computational aspects of continuous wave data analysis and its optimization'', Rome, Italy}

\entrys
{Spring 2014}
{Monographic lecture for graduate students {\it ''Relativistic Astrophysics and Related Computational Methods''} ({\tt https://users.camk.edu.pl/bejger/lectures})} 

\entrys
{2010--2016} 
{{\it Summer@NCAC programme}: supervision of master students on projects related to astrophysics and computational problems (2 each year)} 

\entrys
{2015--} 
{Supervision of theses: PhD - 2, bachelor - 1} 
\end{entrylistshort}


%\pagebreak 

\section{Popularization of science}
\begin{entrylistshort}

\entrys 
{2011--present} 
{Astronomy editor at the ''Delta'' monthly magazine, aimed at the high-school and pre-graduate students interested in mathematics, computer science, physics and astronomy \href{http://www.deltami.edu.pl/delta/autorzy/michal\_bejger}{(in Polish: journal author's website)}} 

\entrys
{see also} 
{Scientific \href{http://users.camk.edu.pl/bejger/outreach/}{outreach} site for the list of texts and recordings} 

\entrys
{2014--present} 
{Polgraw-Virgo Collaboration outreach representative} 
\end{entrylistshort}

\vskip 0.5cm 
% Organization of scientific meetings 
\section{Organization of scientific meetings}

\begin{entrylistshort}

\entrys
{2--5.09.2019} 
{LIGO-Virgo Collaboration meeting, Warsaw, Poland (LOC, 250 participants)} 

\entrys
{26--28.03.2018} 
{POLNS18, Warsaw, Poland (SOC \& LOC, 57 participants)} 

\entrys
{27--31.03.2017} 
{Annual NewCompStar Conference 2017, Warsaw, Poland (SOC \& LOC, 150 participants)} 
 
\entrys
{22--23.10.2012} 
{HyperoNS12 workshop, Warsaw, Poland (LOC, 24 participants)} 

\entrys 
{22--25.09.2010} 
{Joint LIGO-Virgo Meeting, Kraków, Poland (LOC, remote participation system manager, 150 participants)} 

\end{entrylistshort}


\vskip 0.5cm 
% Collaborations and memberships
\section{Collaborations and memberships}

\begin{entrylistshort}

\entrys
{2011--present} 
{Member of the Virgo gravitational-wave detector project and the LIGO-Virgo collaboration}

\entrys
{2013--2017} 
{Polish Einstein Telescope design \& study team}

\entrys 
{2015--present}
{International Astronomical Union} 

\entrys
{2016--present}
{Polish Astronomical Society} 

\end{entrylistshort}

\pagebreak 
% Software projects 
\section{Software projects} 

\begin{entrylistshort}

\entrys
{PolgrawAllSky} 
{Data-analysis pipeline, implementing the network-of-detectors time-domain $\mathcal{F}$-statistic method search for almost monochromatic gravitational wave signals ({\tt https://github.com/mbejger/polgraw-allsky})} 

\entrys 
{SageManifolds} 
{Contribution to the free and open source computer algebra system {\it SageMath} ({\tt http://www.sagemath.org}) with the implementation of the differential geometry and symbolic tensor calculus package {\it SageManifolds} ({\tt http://sagemanifolds.obspm.fr})}  

\end{entrylistshort}

%% Personal details
%\section{Personal details} 
%
%\begin{entrylistshort}
%
%\entrys
%{Nationality}{Polish} 
%
%\entrys
%{Place and date of birth}{Warsaw, Poland, 26 December 1977} 
%
%\entrys
%{Home address}{ul. Wawozowa 28/52, 02-796 Warsaw, Poland} 
% 
%\end{entrylistshort}



%----------------------------------------------------------------------------------------

\end{document}

\printbibsection{article}{article in peer-reviewed journal} % Print all articles from the bibliography

\printbibsection{book}{books} % Print all books from the bibliography

\begin{refsection} % This is a custom heading for those references marked as "inproceedings" but not containing "keyword=france"
\nocite{*}
\printbibliography[sorting=chronological, type=inproceedings, title={international peer-reviewed conferences/proceedings}, notkeyword={france}, heading=bibheading]
\end{refsection}

\begin{refsection} % This is a custom heading for those references marked as "inproceedings" and containing "keyword=france"
\nocite{*}
\printbibliography[sorting=chronological, type=inproceedings, title={local peer-reviewed conferences/proceedings}, keyword={france}, heading=bibheading]
\end{refsection}

\printbibsection{misc}{other publications} % Print all miscellaneous entries from the bibliography

\printbibsection{report}{research reports} % Print all research reports from the bibliography

